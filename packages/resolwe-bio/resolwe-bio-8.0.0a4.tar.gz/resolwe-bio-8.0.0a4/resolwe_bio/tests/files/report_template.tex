\documentclass[11pt]{article}

\pdfobjcompresslevel=0

% Graphics, plotting, images:
\usepackage{graphicx}

% Tweaking text borders:
\usepackage[top=0.89cm, bottom=4cm, left=1.27cm, right=1.31cm]{geometry}
\setlength{\headsep}{0.6cm}

% Making hyperlinks:
\usepackage[colorlinks=true, urlcolor=hyperlinkblue, linkcolor=black]{hyperref}

% Making headers/footers:
\usepackage{fancyhdr}
\setlength{\headheight}{100pt}
\pagestyle{fancy}
\fancyhf{}
\renewcommand{\headrulewidth}{0pt}% Remove header rule
\fancyhead[LE,LO]{\includegraphics[width=35mm]{{#LOGO#}}\\\hrulefill\\~}
\fancyhead[CE,CO]{~\\\hrulefill\\~}
\fancyhead[RE,RO]{ \nouppercase{\leftmark\ | Page \thepage\ of \pageref{LastPage}}\\\hrulefill\\ {\lightfont \fontsize{10pt}{10pt}\selectfont {#SAMPLE_NAME#} | {#PANEL#} | \today}}

% Make only a subset of pages in landscape mode
\usepackage{pdflscape}

% tables spanning multiple pages
\usepackage{longtable}
\usepackage{array}
\newcolumntype{L}{>{\centering\arraybackslash}m{2.3cm}}

\usepackage{booktabs}

%font setup
%\usepackage[sfdefault, medium]{roboto}
\usepackage{helvet}
\renewcommand\familydefault{\sfdefault}
\usepackage[T1]{fontenc}
\newcommand{\lightfont}{\fontseries{l}\selectfont}
\newcommand{\thinfont}{\fontseries{t}\selectfont}
\newcommand{\mediumfont}{\fontseries{m}\selectfont}
\newcommand{\boldfont}{\fontseries{b}\selectfont}

%multiple columns
\usepackage{multicol}
\setlength\columnsep{5mm}

%colors
\usepackage[table]{xcolor}
\definecolor{gray1}{HTML}{EDEDED}
\definecolor{lightblue1}{HTML}{E0F1F5}
\definecolor{darkblue1}{HTML}{2B8196}
\definecolor{hyperlinkblue}{HTML}{1F759B}

%tables
\usepackage{tabularx}
\usepackage{array}
\renewcommand{\arraystretch}{1.5}

\let\oldlongtable\longtable
\let\endoldlongtable\endlongtable
\renewenvironment{longtable}{\rowcolors{2}{lightblue1}{white}\oldlongtable} {
\endoldlongtable} %sets color scheme for long tables

%last page (for page count)
\usepackage{lastpage}

%change of caption styles
\usepackage[font=small]{caption}
\captionsetup{justification=raggedright,singlelinecheck=false}
\captionsetup[table]{ labelfont=it,textfont={it}}
\captionsetup[figure]{labelfont={it}}

%change of chapter styles
\usepackage{titlesec}
\titleformat{\section}
  {\normalfont \fontsize{16pt}{16pt}\selectfont}{\thesection}{1em}{}

\begin{document}

\noindent
{\fontsize{16pt}{16pt}\selectfont \color{darkblue1}{\textbf{{#SAMPLE_NAME#}}}}

\medskip
\noindent
{\lightfont {#PANEL#} | \today}

\medskip
\noindent Results have been filtered to exclude any putative variant detected below AF = {{#AF_THRESHOLD#}}. Variants below this threshold may be inspected manually in the unfiltered VCF files.

\section{QC information}

\begin{multicols}{2}
\noindent
{\color{darkblue1} \fontsize{14pt}{14pt}\selectfont Amplicon performance}\\

\noindent
\lightfont
\rowcolors{2}{white}{gray1}
\begin{tabularx}{\columnwidth}{X l}

Total reads: & \textbf{{#TOTAL_READS#}} \\
Aligned reads: & \textbf{{#ALIGNED_READS#}} \textbf{({#PCT_ALIGNED_READS#})} \\
Aligned bases (on target):& \textbf{{#ALIGN_BASES_ONTARGET#}} \\
Mean coverage:& \textbf{{#COV_MEAN#}} \\
Threshold coverage (20\% of Mean): & \textbf{{#COV_20#}} \\
Coverage uniformity (\% bases covered above 20\% of Mean): & \textbf{{#COV_UNI#}}  \\
\end{tabularx}

\columnbreak

\noindent
\mediumfont
{\color{darkblue1} \fontsize{14pt}{14pt}\selectfont  Per amplicon coverage}\\

\noindent
\lightfont
\rowcolors{2}{gray1}{white}
\begin{tabularx}{\columnwidth}{X l}

No. of amplicons:& \textbf{{{#ALL_AMPLICONS#}}} \\
No. of amplicons with 100\% coverage:& \textbf{{#COVERED_AMPLICONS#} ({#PCT_COV#})} \\
\end{tabularx}
\end{multicols}


{#BAD_AMPLICON_TABLE#}

\newpage
\newgeometry{left=1.23cm, bottom=1.01cm, right=1.27cm, top=3.74cm}
\setlength{\headsep}{0.5cm}

\begin{landscape}
\renewcommand{\arraystretch}{1.4}
    \section{Annotated variants}
    \footnotesize
    {#VCF_TABLES#}
\end{landscape}




\end{document}
